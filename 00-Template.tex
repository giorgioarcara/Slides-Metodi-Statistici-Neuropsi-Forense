% Options for packages loaded elsewhere
% Options for packages loaded elsewhere
\PassOptionsToPackage{unicode}{hyperref}
\PassOptionsToPackage{hyphens}{url}
%
\documentclass[
  ignorenonframetext,
]{beamer}
\newif\ifbibliography
\usepackage{pgfpages}
\setbeamertemplate{caption}[numbered]
\setbeamertemplate{caption label separator}{: }
\setbeamercolor{caption name}{fg=normal text.fg}
\beamertemplatenavigationsymbolsempty
% remove section numbering
\setbeamertemplate{part page}{
  \centering
  \begin{beamercolorbox}[sep=16pt,center]{part title}
    \usebeamerfont{part title}\insertpart\par
  \end{beamercolorbox}
}
\setbeamertemplate{section page}{
  \centering
  \begin{beamercolorbox}[sep=12pt,center]{section title}
    \usebeamerfont{section title}\insertsection\par
  \end{beamercolorbox}
}
\setbeamertemplate{subsection page}{
  \centering
  \begin{beamercolorbox}[sep=8pt,center]{subsection title}
    \usebeamerfont{subsection title}\insertsubsection\par
  \end{beamercolorbox}
}
% Prevent slide breaks in the middle of a paragraph
\widowpenalties 1 10000
\raggedbottom
\AtBeginPart{
  \frame{\partpage}
}
\AtBeginSection{
  \ifbibliography
  \else
    \frame{\sectionpage}
  \fi
}
\AtBeginSubsection{
  \frame{\subsectionpage}
}
\usepackage{iftex}
\ifPDFTeX
  \usepackage[T1]{fontenc}
  \usepackage[utf8]{inputenc}
  \usepackage{textcomp} % provide euro and other symbols
\else % if luatex or xetex
  \usepackage{unicode-math} % this also loads fontspec
  \defaultfontfeatures{Scale=MatchLowercase}
  \defaultfontfeatures[\rmfamily]{Ligatures=TeX,Scale=1}
\fi
\usepackage{lmodern}

\usetheme[]{metropolis}
\ifPDFTeX\else
  % xetex/luatex font selection
\fi
% Use upquote if available, for straight quotes in verbatim environments
\IfFileExists{upquote.sty}{\usepackage{upquote}}{}
\IfFileExists{microtype.sty}{% use microtype if available
  \usepackage[]{microtype}
  \UseMicrotypeSet[protrusion]{basicmath} % disable protrusion for tt fonts
}{}
\makeatletter
\@ifundefined{KOMAClassName}{% if non-KOMA class
  \IfFileExists{parskip.sty}{%
    \usepackage{parskip}
  }{% else
    \setlength{\parindent}{0pt}
    \setlength{\parskip}{6pt plus 2pt minus 1pt}}
}{% if KOMA class
  \KOMAoptions{parskip=half}}
\makeatother


\usepackage{longtable,booktabs,array}
\usepackage{calc} % for calculating minipage widths
\usepackage{caption}
% Make caption package work with longtable
\makeatletter
\def\fnum@table{\tablename~\thetable}
\makeatother
\usepackage{graphicx}
\makeatletter
\newsavebox\pandoc@box
\newcommand*\pandocbounded[1]{% scales image to fit in text height/width
  \sbox\pandoc@box{#1}%
  \Gscale@div\@tempa{\textheight}{\dimexpr\ht\pandoc@box+\dp\pandoc@box\relax}%
  \Gscale@div\@tempb{\linewidth}{\wd\pandoc@box}%
  \ifdim\@tempb\p@<\@tempa\p@\let\@tempa\@tempb\fi% select the smaller of both
  \ifdim\@tempa\p@<\p@\scalebox{\@tempa}{\usebox\pandoc@box}%
  \else\usebox{\pandoc@box}%
  \fi%
}
% Set default figure placement to htbp
\def\fps@figure{htbp}
\makeatother





\setlength{\emergencystretch}{3em} % prevent overfull lines

\providecommand{\tightlist}{%
  \setlength{\itemsep}{0pt}\setlength{\parskip}{0pt}}



 


\setbeamerfont{title}{size=\large} \usepackage{hyperref} \setbeamertemplate{footline}[frame number]
\makeatletter
\@ifpackageloaded{caption}{}{\usepackage{caption}}
\AtBeginDocument{%
\ifdefined\contentsname
  \renewcommand*\contentsname{Table of contents}
\else
  \newcommand\contentsname{Table of contents}
\fi
\ifdefined\listfigurename
  \renewcommand*\listfigurename{List of Figures}
\else
  \newcommand\listfigurename{List of Figures}
\fi
\ifdefined\listtablename
  \renewcommand*\listtablename{List of Tables}
\else
  \newcommand\listtablename{List of Tables}
\fi
\ifdefined\figurename
  \renewcommand*\figurename{Figure}
\else
  \newcommand\figurename{Figure}
\fi
\ifdefined\tablename
  \renewcommand*\tablename{Table}
\else
  \newcommand\tablename{Table}
\fi
}
\@ifpackageloaded{float}{}{\usepackage{float}}
\floatstyle{ruled}
\@ifundefined{c@chapter}{\newfloat{codelisting}{h}{lop}}{\newfloat{codelisting}{h}{lop}[chapter]}
\floatname{codelisting}{Listing}
\newcommand*\listoflistings{\listof{codelisting}{List of Listings}}
\makeatother
\makeatletter
\makeatother
\makeatletter
\@ifpackageloaded{caption}{}{\usepackage{caption}}
\@ifpackageloaded{subcaption}{}{\usepackage{subcaption}}
\makeatother

\usepackage{bookmark}
\IfFileExists{xurl.sty}{\usepackage{xurl}}{} % add URL line breaks if available
\urlstyle{same}
\hypersetup{
  pdfauthor={Giorgio Arcara},
  hidelinks,
  pdfcreator={LaTeX via pandoc}}


\author{Giorgio Arcara}
\date{}

\begin{document}


\begin{frame}
% TITLE SLIDES
\title{Metodi Statistici per la Neuropsicologia Forense\\ \vspace{1em} \emph{Introduzione}}
\author{Giorgio Arcara,\\ Università di Padova \\ IRCCS San Camillo, Venezia}

\titlegraphic{

\vspace*{7cm}
\includegraphics[scale=0.15]{Figures/LogoSanCamilloIRCCS_Unipd_alpha.png}
\begin{center}
\vspace{-2.2em}
\includegraphics[scale=0.15]{Figures/CC_license_3_0.png}
\end{center}
}
\date{\today}
\maketitle
\end{frame}

\begin{frame}{Prima slide}
\phantomsection\label{prima-slide}
\begin{itemize}
\tightlist
\item
  Presentazione del docente e del corso.
\item
  Aspetti organizzativi del corso.
\item
  Obiettivi formativi e ``spirito'' del corso.
\end{itemize}
\end{frame}

\begin{frame}{Seconda slides (due colonne)}
\phantomsection\label{seconda-slides-due-colonne}
\begin{columns}
\column{0.8\textwidth}
\small
\textbf{L'aspetto da ricordare è che nei test neuropsicologici, il più delle volte in maniera implicita, è assunta la definizione di misurazione di Stevens e che sia possibile assumere una misura in una scala ad intervalli.} 
\column{0.2\textwidth}

\begin{figure}
\includegraphics[scale=0.05]{Figures/triangle.png}
\end{figure}
\end{columns}
\end{frame}

\begin{frame}{slide 3 figura e box}
\phantomsection\label{slide-3-figura-e-box}

\begin{tikzpicture}
    % Include your image
    \node[anchor=south west, inner sep=0] (image) at (0,0)
        {\includegraphics[width=0.8\textwidth]{Figures/Test_Methodology_transp.png}};

    % Set coordinate system to match the image
    \begin{scope}[x={(image.south east)}, y={(image.north west)}]
        
        % Draw a red rectangle (thin border)
        %\draw[red, thick, rounded corners] (0.25,0.3) rectangle (0.45,0.5);
        % Optional label
        %\node[red, font=\bfseries, above] at (0.35,0.5) {Region 1};
        
        %\pause
        % a red rectangle (thicker border)
        \draw[red, ultra thick] (0.605,0.65) rectangle (0.905,0.85);
        
        % text (with fine control)
        \node[anchor=west, xshift=-8mm, yshift=20mm] at (image.east)
        {\parbox{4cm}{\parbox{3cm}{\scriptsize \textbf{Minacce alla validità}\\ (il test non misura quello che vorrebbe misurare)}}};


        
    \end{scope}
\end{tikzpicture}
\end{frame}




\end{document}
